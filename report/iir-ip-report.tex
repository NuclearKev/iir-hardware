\documentclass[12pt,a4paper,titlepage]{article}

\usepackage{times}
\usepackage{setspace}
\usepackage{textcomp}
\usepackage[a4paper,left=1in,top=1in,right=1in,bottom=1in]{geometry}

\usepackage[english]{babel}
\usepackage[backend=biber]{biblatex}
\usepackage{csquotes}

\usepackage[bookmarks]{hyperref}
\hypersetup{colorlinks=true,allcolors=black}
\usepackage{hypcap}


\bibliography{sources.bib}


\doublespacing{}
\oddsidemargin = 0pt
\headheight = 0pt
\topmargin = 0pt
\headsep = 0pt
\marginparwidth = 0pt
\textwidth = 450pt


\begin{document}

\title{IIR IP}
\author{Kevin Bloom \\ Rev A}
\maketitle

\newpage

\tableofcontents

\newpage

\section{Introduction}
When selecting this project, I was under the assumption that it would be fairly
easy to complete. Seeing as completing this task in software isn't a huge deal,
I assumed the same for hardware. To my surprise, this was far from true. The
calculation for IIR is quite difficult to complete in hardware, as I will
discuss in this report. I will be discussing the theory behind the IIR in
hardware, the major issues that can occur, and how one could complete this task
in the future.

\subsection{Important Files}
Inside of this project, there are a few different files that are important.
Firstly, my notebook. This file is entitled \texttt{notebook.org} and contains
information on my process throughout the semester. You can open it with a text
editor of choice, or you can view the exported \texttt{notebook.html}
file. Please note that the HTML version doesn't contain all of the clock
stamps. Inside of \texttt{presentation/} is the presentation and its source. The
source for this report can be found in \texttt{report/}. Inside of
\texttt{projects/} will be all of the different projects that I worked on. The
important ones to note are: \texttt{2nd-order-single-section/},
\texttt{complex-iir/}, and \texttt{the-really-big-one/}.

Just so that it's easier to find your way, I will do a quick description of each
project. This will prevent you from having to search
around. \texttt{2nd-order-single-section/} is a design that only uses a single
section BiQuad. This was used to prove that there was something wrong with the
BiQuad implementation of the IP. \texttt{complex-iir/} contains a design that
doesn't use the Zynq. It does everything in hardware. The current setup uses
internally selected coefficients, opposed to the normally externally fed
coefficients. Lastly, \texttt{the-really-big-one/} is a design that uses pretty
much everything. It is set up with 4 BiQuads cascaded serially and uses the dmux
and mux IPs to make the design generic.

\section{Single-Section}


\subsection{BiQuad}

\appendix

\newpage

\printbibliography[
heading=bibintoc,
title={Resources}
]

\end{document}
